\documentclass[12pt,t]{beamer}
\usepackage{graphicx}
\setbeameroption{hide notes}
\setbeamertemplate{note page}[plain]
\usepackage{listings}

% get rid of junk
\usetheme{default}
\beamertemplatenavigationsymbolsempty
\hypersetup{pdfpagemode=UseNone} % don't show bookmarks on initial view


% font
\usepackage{fontspec}
\setsansfont
  [ ExternalLocation = ../fonts/ ,
    UprightFont = *-regular , 
    BoldFont = *-bold ,
    ItalicFont = *-italic ,
    BoldItalicFont = *-bolditalic ]{texgyreheros}
\setbeamerfont{note page}{family*=pplx,size=\footnotesize} % Palatino for notes
% "TeX Gyre Heros can be used as a replacement for Helvetica"
% I've placed them in ../fonts/; alternatively you can install them
% permanently on your system as follows:
%     Download http://www.gust.org.pl/projects/e-foundry/tex-gyre/heros/qhv2.004otf.zip
%     In Unix, unzip it into ~/.fonts
%     In Mac, unzip it, double-click the .otf files, and install using "FontBook"

% named colors
\definecolor{offwhite}{RGB}{255,250,240}
\definecolor{gray}{RGB}{155,155,155}

\ifx\notescolors\undefined % slides
  \definecolor{foreground}{RGB}{255,255,255}
  \definecolor{background}{RGB}{24,24,24}
  \definecolor{title}{RGB}{107,174,214}
  \definecolor{subtitle}{RGB}{102,255,204}
  \definecolor{hilit}{RGB}{102,255,204}
  \definecolor{vhilit}{RGB}{255,111,207}
  \definecolor{lolit}{RGB}{155,155,155}
\else % notes
  \definecolor{background}{RGB}{255,255,255}
  \definecolor{foreground}{RGB}{24,24,24}
  \definecolor{title}{RGB}{27,94,134}
  \definecolor{subtitle}{RGB}{22,175,124}
  \definecolor{hilit}{RGB}{122,0,128}
  \definecolor{vhilit}{RGB}{255,0,128}
  \definecolor{lolit}{RGB}{95,95,95}
\fi
\definecolor{nhilit}{RGB}{128,0,128}  % hilit color in notes
\definecolor{nvhilit}{RGB}{255,0,128} % vhilit for notes

\newcommand{\hilit}{\color{hilit}}
\newcommand{\vhilit}{\color{vhilit}}
\newcommand{\nhilit}{\color{nhilit}}
\newcommand{\nvhilit}{\color{nvhilit}}
\newcommand{\lolit}{\color{lolit}}

% use those colors
\setbeamercolor{titlelike}{fg=title}
\setbeamercolor{subtitle}{fg=subtitle}
\setbeamercolor{institute}{fg=lolit}
\setbeamercolor{normal text}{fg=foreground,bg=background}
\setbeamercolor{item}{fg=foreground} % color of bullets
\setbeamercolor{subitem}{fg=lolit}
\setbeamercolor{itemize/enumerate subbody}{fg=lolit}
\setbeamertemplate{itemize subitem}{{\textendash}}
\setbeamerfont{itemize/enumerate subbody}{size=\footnotesize}
\setbeamerfont{itemize/enumerate subitem}{size=\footnotesize}

% page number
\setbeamertemplate{footline}{%
    \raisebox{5pt}{\makebox[\paperwidth]{\hfill\makebox[20pt]{\lolit
          \scriptsize\insertframenumber}}}\hspace*{5pt}}

% add a bit of space at the top of the notes page
\addtobeamertemplate{note page}{\setlength{\parskip}{12pt}}

% default link color
\hypersetup{colorlinks, urlcolor={hilit}}

\ifx\notescolors\undefined % slides
  % set up listing environment
  \lstset{language=bash,
          basicstyle=\ttfamily\scriptsize,
          frame=single,
          commentstyle=,
          backgroundcolor=\color{darkgray},
          showspaces=false,
          showstringspaces=false
          }
\else % notes
  \lstset{language=bash,
          basicstyle=\ttfamily\scriptsize,
          frame=single,
          commentstyle=,
          backgroundcolor=\color{offwhite},
          showspaces=false,
          showstringspaces=false
          }
\fi

% a few macros
\newcommand{\bi}{\begin{itemize}}
\newcommand{\bbi}{\vspace{24pt} \begin{itemize} \itemsep8pt}
\newcommand{\ei}{\end{itemize}}
\newcommand{\ig}{\includegraphics}
\newcommand{\subt}[1]{{\footnotesize \color{subtitle} {#1}}}
\newcommand{\ttsm}{\tt \small}
\newcommand{\ttfn}{\tt \footnotesize}
\newcommand{\figh}[2]{\centerline{\includegraphics[height=#2\textheight]{#1}}}
\newcommand{\figw}[2]{\centerline{\includegraphics[width=#2\textwidth]{#1}}}



%%%%%%%%%%%%%%%%%%%%%%%%%%%%%%%%%%%%%%%%%%%%%%%%%%%%%%%%%%%%%%%%%%%%%%
% end of header
%%%%%%%%%%%%%%%%%%%%%%%%%%%%%%%%%%%%%%%%%%%%%%%%%%%%%%%%%%%%%%%%%%%%%%

% title info
\title{12: Summary + Extras}
\author{}
\date{\href{https://bit.ly/SISBID3}{\tt \color{foreground} bit.ly/SISBID3}}

\begin{document}

% title slide
{
\setbeamertemplate{footline}{} % no page number here
\frame{
  \titlepage
  \note{We'll briefly summarize all that we discussed, and further
    touch on how to share your research data and code. Then we'll give
    pointers to the many things that we didn't have time to talk about.
} }


\begin{frame}[c]{}

\begin{center}
\large
The most important tool is the {\hilit mindset},\\
when starting, that the end product \\
will be reproducible.
\end{center}

\hfill
{\lolit
{\textendash} \href{http://odin.mdacc.tmc.edu/~kabaggerly/}{Keith Baggerly}
}

\note{So true. Desire for reproducibility is step one.
}
\end{frame}




\begin{frame}[c]{Steps toward reproducible research}


  \bi
\item Slow down
\item Organize; document
\item Everything with code
\item Scripts → RMarkdown
\item Code → functions → packages
\item Version control with Git
\item Automation with Make
\item Choose a license
\item Share your work with others
  \ei


\note{Moving from ``standard practice'' to ``fully reproducible'' is
  hard. There are a lot of tools to learn and a lot of workflow
  changes to make. Don't try to change everything all at once. Focus
  on improving one aspect at a time, ideally jointly with your friends
  and colleagues. Your goal should be to have each project be a bit
  better organized than the previous.
}
\end{frame}



\begin{frame}[c]{Challenges}


  \bi
\item Daily maintenance
  \bi
\item READMEs up to date?
\item Documentation match code?
  \ei
\item Cleaning up the junk
  \bi
\item Move defunct stuff into an {\tt Old/} subdirectory?
  \ei
\item Start over from the beginning, nicely?
  \ei


\note{The organization of a project is dependent on your worst day
  with it. You keep everything carefully arranged for months and then
  one day you need to rush, rush, rush to get a manuscript out the
  door, and you leave a big mess.

  And a common problem is that you don't really know what you're doing
  until it's all done. So if you can, spend a week at the end making a
  separate, clean version of the whole thing.
}
\end{frame}



\begin{frame}[c]{Sharing your work}

  \bi
\item Why share?
  \bi
\item Funding agency or journal requirement
\item Increased visibility
\item So that others can build on your work
  \ei
\item When?
  \bi
\item Continuously and instantaneously
\item When you submit a paper
\item When your paper appears
  \ei
\item Risks?
  \ei

  \note{
    There a lot of advantages to making your work public, and you may
    be required to make it public.

    Some scientists put all of the work in the open as they're doing
    it. Others wait until they submit a manuscript; others until the
    manuscript actually appears. Still others want to delay releasing
    data yet further. The earlier the better, I think.

    Are there risks? Many worry about being scooped: having someone
    find something cool in your data before you got a chance to find
    it yourself. I think the risk of this is far outweighed by the
    advantages of having data and code in the open. You may also worry
    about people finding problems in your analyses. But if there are
    problems, wouldn't you rather know about them? You can't build
    upon it if it's not right.

    The biggest risk is that you'll be ignored. Sharing more,
    sooner, and in a more convenient form will encourage broader
    visibility.
  }


\end{frame}


\begin{frame}[c]{What to share?}

  \bi
\item For sure
  \bi
\item Primary dataset
\item Metadata
\item Data cleaning scripts
\item Analysis scripts
  \ei
\item It could help
  \bi
\item Very-raw data
\item Processed/clean data
\item Intermediate results
  \ei
\item No
  \bi
\item Confidential data (e.g. HIPAA data)
\item Passwords, private keys
  \ei
  \ei

  \note{It can be tricky to define what is the ``primary dataset''.
    How raw of data should you share? Where would someone want to
    start, and how can you make it easiest for them to dig into your
    analyses or make use of the data for other purposes, such as for a
    meta-analysis study? In some cases, it can be valuable to share
    many of the intermediate results, so that folks can jump into the
    bit that they care about without having to first run a bunch of
    complex and time-consuming analyses to get there.

    You definitely want to make sure that you're not including any
    confidential information, particularly regarding patient-level
    data, but also private keys for data APIs which you might be
    including in your scripts.
}
\end{frame}



\begin{frame}[c]{Where to share?}

  \bi
\item Domain-specific repository
  \bi
  \item Genbank, dbGaP, etc.
  \item See \href{http://re3data.org}{\tt re3data.org}
  \ei
\item \href{https://figshare.com}{Figshare}, \href{https://datadryad.org}{Dryad}, \href{https://zenodo.org}{Zenodo}
\item Institutional repository
  \item \href{https://github.com}{GitHub}, \href{https://bitbucket.org}{BitBucket}
  \ei

  \note{Share code at GitHub or BitBucket. But how about data?
    Consider domain-specific repositories like dbGaP, then general
    data repositories like Zenodo, and then look to see whether your
    institution has a data repository.

    Another option is as supplemental material for a publication, at
    the Journal's website, but this is often the most cumbersome for
    users.}
\end{frame}


\begin{frame}[c]{Resources}

  \bi
\item R Markdown
  \bi
    \item \href{https://rmarkdown.rstudio.com}{\tt rmarkdown.rstudio.com}
  \ei
\item R Packages
  \bi
    \item Releasing to CRAN:
      \href{http://r-pkgs.had.co.nz/release.html}{\tt \footnotesize r-pkgs.had.co.nz/release.html}
    \item Leek group: \href{https://github.com/jtleek/rpackages}{\tt
      \footnotesize github.com/jtleek/rpackages}
    \item When to trust an R package:
      \href{https://simplystatistics.org/2015/11/06/how-i-decide-when-to-trust-an-r-package/}{\tt
        \footnotesize bit.ly/trust\_r\_pkg}
  \ei
\item Make
  \bi
    \item \href{http://kbroman.org/minimal_make}{\tt kbroman.org/minimal\_make}
  \ei
\item Git
  \bi
    \item Git branches:
      \href{https://nicercode.github.io/git/branches.html}{\tt
        \footnotesize nicercode.github.io/git/branches.html}
    \item Hadley on Git/GitHub:
      \href{http://r-pkgs.had.co.nz/git.html}{\tt \footnotesize r-pkgs.had.co.nz/git.html}
    \item Git subtrees:
      \href{https://developer.atlassian.com/blog/2015/05/the-power-of-git-subtree/}{\tt
        \footnotesize bit.ly/git\_subtree}
  \ei
  \ei

  \note{ Links to some resources related to the course material. }
\end{frame}


\begin{frame}[c]{Some of the things we didn't cover}

  \bi
\item Command-line, including `R CMD BATCH`
\item Software testing (and debugging)
\item Code review / paired programming
\item Capturing versions of dependent software (e.g.
  \href{https://rstudio.github.io/packrat}{packrat})
\item Containers (e.g. \href{https://www.docker.com}{\tt \footnotesize docker.com})
\item Coding conventions, e.g.
  Hadley's \href{http://adv-r.had.co.nz/Style.html}{\tt \footnotesize
    adv-r.had.co.nz/Style.html} and
  Google's
  \href{https://google.github.io/styleguide/Rguide.xml}{\tt
      \footnotesize google.github.io/styleguide/Rguide.xml}
\item Report templates,
  \href{http://rmarkdown.rstudio.com/developer_document_templates.html}{\tt
    \footnotesize rmarkdown.rstudio.com/developer\_document\_templates.html}

  \ei

\note{ There's a lot that we didn't cover.}

\end{frame}

\end{document}
